\documentclass{acm_proc_article-sp}

\usepackage{bussproofs}
\usepackage{amssymb}
\usepackage{amssymb}

\begin{document}

\title{Higher Order and Distributed Intents for the Android platform.}

\author{
Kristopher Micinski\\
       \email{micinski@cs.umd.edu}
\and
Brianna Ren\\
       \email{bren@cs.umd.edu}
}

\maketitle

\begin{abstract}
  The Android platform exhibits a service oriented architecture where
  apps can interact with each other through message passing.  These
  messages --- called intents --- use late binding to allow any number
  of apps to handle them, based on a filter which is used to decide
  which app will respond to the intent.  This gives rise to a
  programming style based on late binding.  We formalize these
  intents, and write down types for each intent, adding higher order
  intents, and a programming style based on Continuation Passing Style
  (CPS).  We also add inter device intents, allowing multiple devices
  to communicate and accomplish a task, even if the tools to
  accomplish the task do not exist (together) on any single device.
  We have a working implementation based on a combination of an
  Android library and a Rails web service (which does intent routing
  for us).  To demonstrate our system --- and as a case study in
  higher order distributed programming --- we design a system that
  allows distributed news collection, where a user can create
  elaborate context based tags on information, and ask that it be
  collected by a number of devices within a region.
\end{abstract}

\section{Introduction}
Android applications exhibit a novel programming model: a service
oriented architecture with message passing between applications to
allow cooperation.  As an example, we might have an application which
allows a user to take a picture, manipulate it with a photo editor (to
remove red eye, for example), and then send it to their friends via
email.  The messages sent between applications are called Intents, and
form the basis for service oriented communication.  In this paper we
present:

\begin{itemize}
\item A formalization of intents, and the Android activity stack.
  This allows precise reasoning about how intents will be handled and
  allows an interpretation of exactly what in intent \emph{is}.
\item A dynamic semantics for an intent languge.
\item From this, a \emph{type} for intents, which allows applications
  to say exactly what type of data they will consume 
\end{itemize}

We first investigate an operational semantics for a scripting language
which allows scripting Android apps out of services provided by each
app.  For example, the camera app will allow a user to take a picture,
and return the bitmap containing that image.  A file chooser app might
allow a user to choose a file and then return the name of the file
that was choosen.  However, we present a small scripting langauge (in
the style of Scheme) that allows users to write scripted programs
which combine this behavior.  While each of the examples given up to
this point was first order (an atomic ``action'' was performed), we
then add the possibility of higher order functions: this allows
programming in a CPS style.  For example, we might write the following
script which allowed the user to execute the following script:

\begin{verbatim}
(send-email (choose-file
\end{verbatim}

\section{Higher Order Intents}

\section{Android Scheme}

In this section we present Android scheme, a small language which we
have invented to enable scripting with (and demonstrate the power of)
android intents.  A user (who wants to write Android apps, out of
existing Android components) will write their app using Android
scheme, and then pass those programs to the Android scheme
interpreter, running on a device (this is a standard app, with some
boilerplate code).

Each intent is handled \emph{within} its handling app, and only the
result of the computation leaves the intent, but this is no problem:
we can simply write programs in Android scheme, and then have the
indivual intents call back to the interpreter when they finish, using
a continuation based style.  Android already encourages this
programming style, as when intents finish the can start new intents by
simply passing them to the system.

\section{Operational Semantics}

We base our

\section{Inter-Device Intents}

\section{Case Study: Distributed News Collection}

\section{Related Work}
Oleg \cite{kiselyov:2006} ...

\section{Conclusion}

\bibliography{intents}{}
\bibliographystyle{plain}

\end{document}
